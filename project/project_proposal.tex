\documentclass{article}

\usepackage[latin1]{inputenc}
\usepackage{times,fullpage,amsmath}
\usepackage{enumerate,graphicx,hyperref,verbatim, amsmath, mathtools,pdfpages}
\DeclareMathSizes{10}{10}{10}{10}

\title{Machine Learning on Intrusion detection systems}
\author{Erik Steggall\\ esteggall@soe.ucsc.edu}

\date{Fall 2014}
\setlength\parindent{0pt}
\begin{document} \maketitle \pagestyle{empty}

\section*{My proposal}
I plan to write a survey on the use various machine learning models to improve that accuracy of both network and host based IDS's. To prepare for the writeup I'm going to read the 10 papers listed below. My goal is to try to implement a small example of my favorite model, but I want to be able to use the paper as a fallback in case I'm not able to find a dataset or complete the implementation.\\ 
As of now, I'd like to consider the paper to be the actual graded part of my final project, and any other work I can do will just be additional as I realize I may not even get around to starting it until after the quarter has ended. I have read through chapter 8 of the textbook in order to get a better sense of intrusion detection.\\

\section*{Week 6}
I plan on reading the following papers to get a better understanding of both intrusion detection systems, as well as how machine learning is being used to improve current techniques. The following papers are introductory papers to intrusion detection, and how machine learning is applied to IDS to improve performance:\\

This article looks like a good overview of Intrusion Detection design.\\
Intrusion detection techniques and approaches, by Theuns Verwoerd, Ray Hunt.\\

This article is an overview of machine learning techniques used for intrusion detection systems:\\
Intrusion detection by machine learning: A review, by Chih-Fong Tsai, Yu-Feng Hsu, Chia-Ying Lin, Wei-Yang Lin.\\

This article hopefully will give more insight on what the main goals of machine learning are for improving IDSs\\
Data mining and machine learning. Towards reducing false positives in intrusion detection, by Tadeusz Pietraszek, Axel Tanner\\

This article goes over two schools of machine learning, supervised and unsupervised. I'm curious to read this because my current understanding is that IDS techniques are based on unsupervised learning.\\
Learning Intrusion Detection: Supervised or Unsupervised, by Pavel Laskov, Patrick Du?ssel, Christin Sch?afer, and Konrad Rieck.\\

\section*{Week 7}
I plan on reading over more detailed articles in this week, I found two articles on host based IDSs, and two on network based IDSs:\\

First host based:\\
Intrusion Detection Applying Machine Learning to Solaris Audit Data, by David Endler.\\

Second host based:\\
LearningPatterns from Unix Process Execution Traces for Intrusion Detection, by WenkeLee and Saivatore J. Stolfo\\

First network based:\\
Outside the Closed World: On Using Machine Learning For Network Intrusion Detection, by Robin Sommer and Vern Paxson\\

Second network based, this one will also hopefully give me insight as to how I might go about implementing my own machine learning techniques, by Chris Sinclair, Lyn Pierce and Sara Matzner\\
An Application of Machine Learning to Network Intrusion Detection.\\

\section*{Week 8}
In week 8 I hope to get the first draft of my paper written, as well as hopefully read the following two papers:\\

Another application based paper, hopefully will help with my implementation ideas:\\
Application of Machine Learning Algorithms to KDD Intrusion Detection Dataset within Misuse Detection Context, by Maheshkumar Sabhnani and Gursel Serpen.\\

This paper goes over more advanced concepts of machine learning and should be a fun read:\\
Intrusion Detection Using Neural Networks and Support Vector Machines, by Srinivas Mukkamala, Guadalupe Janoski, Andrew Sung\\

\section*{Week 9}
Work on finalizing my papers draft, hopefully start implementation here too.\\

\section*{Week 10}
Again, I want to have a lot of buffer room, so I'll leave this week open, but hopefully implementation of a small subset of one of the concepts that grabs me from the papers I read.\\

\end{document}